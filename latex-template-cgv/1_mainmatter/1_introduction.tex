\iflanguage{ngerman}
{\chapter{Einleitung}}
{\chapter{Introduction}}

\label{sec:introduction}
The chapter should broadly contextualize your research and motivate your work.

\section{Motivation}
In today's increasingly intelligence-focused world, many devices we once regarded as mechanical, fixed, and lacking adaptability are now seeking new changes. Robotics, having been adapted in industries for decades are predominantly used under fixed and pre-defined environment. All the moves are implanted before the actual operation so it has few capabilities reacting to the change of environment. It was partly because of the fact that there are very few sensors applicable for sensing the enviornment. However, with the increasing availability of various sensors on the market, different solutions have made it possible to develop intelligent systems with interactive capabilities, environmental awareness, and adaptive abilities.

DIGIT \cite{Mike2020} designed as textile sensor can be used for robotics that requires high-resouluted texture information. With the help of it, robot arm can detect small deformation of textile surface of contact object which can help increase the precision of robot motion planning. Handling small components, especially in the sense of laboratory environment, needs extra accuracy compared to industrie robots. The modular componentes robot arm dealing with in laboratory does not have a regular shape which means the grasp of robot grippers is not always stable, components might slide away while grasping. Also it happens at the moment of attachment and detachment. They all together raise challenge for intefrating DIGIT sensor in the current system.

\section{Goal}
This section should clarify, what should be achieved by the work.

\section{Structure of the Work}
Here the structure can be \emph{briefly} explained.


